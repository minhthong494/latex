\documentclass[12pt,a4paper,openany,oneside]{report}
\usepackage[utf8]{vietnam}
\usepackage{amsmath, amsthm, amssymb,amsxtra,latexsym,amscd,graphicx,graphpap,makeidx}
\usepackage{array,tabularx,longtable,multicol,indentfirst}%
\usepackage[mathscr]{eucal}
\usepackage[top=3.5cm, bottom=3.0cm, left=3.5cm, right=2cm] {geometry}
%========================================================================
%--Dat ten-----------------------------------
\newtheorem{dl}{Định lý}[chapter]
\newtheorem{dn}{Định nghĩa}[chapter]
\newtheorem{bt}{Bài toán}[chapter]
\newtheorem{bta}{Bài}[chapter]
\newtheorem{tc}{Tính chất}[chapter]
\newtheorem{md}{Mệnh đề}[chapter]
\newtheorem{bd}{Bổ đề}[chapter]
\newtheorem{hq}{Hệ quả}[chapter]
\newtheorem{nx}{Nhận xét}[chapter]
\newtheorem{cy}{Chú ý}[chapter]
\newtheorem{vd}{Ví dụ}[chapter]
\renewcommand{\chaptername}{Chương}
\renewcommand\bibname{Tài liệu tham khảo}
%-----------------------------------------------
\def\en{\enskip}
\def\n{\noindent}
\def\m{\medskip}
\def\en{\enskip}
\def\m{\medskip}
\def\n{\noindent}
\def\Re{\mbox{Re }}
\def\Im{\mbox{Im }}
\def\hcm{\hfill $\square$\\}
\def\imotn{i = 1, 2, \ldots, n}
\def\ii{\item}
%-----------------------------------
\begin{document}
\fontsize{14pt}{14pt}\selectfont \baselineskip 0.65cm
\thispagestyle{empty}
\begin{center}
\MakeUppercase{{\bf\small Bộ giáo dục và Đào tạo}}\\
\MakeUppercase{{\bf\small Trường ..............}}\\
\vspace{2cm}
\MakeUppercase{\textsc{{\large\bf  nguoicontraiphonui.blogspot.com}}}\\
\vspace{1.8cm}
%\begin{center}
\textbf{\MakeUppercase{\textsc{\huge\bf Tên đề tài dòng 1}}}\\
\textbf{\MakeUppercase{\textsc{\huge\bf tiếp tên đề tài dòng 2}}}\\
%\end{center}
\vspace{2.5cm }
\MakeUppercase{\textsc{{\bf \small  Luận văn Thạc sỹ}}}\\
\vspace{2.0cm }
{\large \bf Chuyên ngành :\textbf{\MakeUppercase{\textsc{\small{\,\,\, gõ chuyên ngành vào đây}}}}}\\
%\ftmhai

{\large\bf  Mã số} : {{\bf\large   00\hspace*{0.25cm}00\hspace*{0.15cm}00}}\\
\end{center}
\vspace{0.5cm}
\begin{center}
{\large\bf Giáo viên hướng dẫn}:\\
\MakeUppercase{\textsc{{\bf\small chức danh và tên giáo viên hướng dẫn}}}\\
\vspace{1.8cm}
%\begin{center}
\MakeUppercase{\textsc{{\footnotesize \bf Thành phố, 2012}}}
\end{center}

\tableofcontents
\newpage

\chapter*{Mở đầu}
\addcontentsline{toc}{chapter}{\quad\  \bf Mở đầu}

\n \MakeUppercase{\bf 1. Lý do chọn đề tài}

\n gõ vào đây
\m

\m


\n \MakeUppercase{\bf 2. Mục đích nghiên cứu}

\n gõ vào đây
\m
\m

\n \MakeUppercase{\bf 3. Đối tượng và phạm vi nghiên cứu}

\n gõ vào đây
\m 

\n \MakeUppercase{\bf 4. Phương pháp nghiên cứu}

\n gõ vào đây
\m

\n \MakeUppercase{\bf 5. Ý nghĩa khoa học và thực tiễn của đề tài}

\n gõ vào đây

\bigskip

\n \MakeUppercase{\bf 6. Cấu trúc của luận văn}
\m

gõ vào đây

%-----------------------------------

\chapter{Chương thứ 1}
\section{Đề mục 1}
\begin{dl}
gõ định lý 1
\begin{equation}%\label{eq:so1} --- đánh dấu cuối công thức
123456789-1=123456788
\end{equation}
\begin{proof}%--chứng minh
ta lấy que tính ra đếm để chỉ ra \eqref{eq:so1} là đúng.
\end{proof}
\end{dl}

%-----
\chapter{Chương thứ 2}

%-----
\chapter{Chương thứ 3}

%-----
\chapter{Chương thứ 4}

%-----
\chapter*{Kết luận}
\addcontentsline{toc}{chapter}{\quad\ \bf Kết luận}

%-----
\begin{thebibliography}{12}
\addcontentsline{toc}{chapter}{\quad\  \bf Tài liệu tham khảo}
\bibitem{1}Họ và tên tác giả, năm, {\it Tên sách} NXB.
\bibitem{2}Họ và tên tác giả, năm, {\it Tên sách} NXB.
\bibitem{3}Họ và tên tác giả, năm, {\it Tên sách} NXB.
\bibitem{4}Họ và tên tác giả, năm, {\it Tên sách} NXB.
\bibitem{5}Họ và tên tác giả, năm, {\it Tên sách} NXB.
\end{thebibliography}
%\tableofcontents
\end{document}
