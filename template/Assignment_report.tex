\documentclass[a4paper]{report}
\usepackage{vntex}
%\usepackage[english,vietnam]{babel}
%\usepackage[utf8]{inputenc}

%\usepackage[utf8]{inputenc}
%\usepackage[francais]{babel}
\usepackage{a4wide,amssymb,epsfig,latexsym,multicol,array,hhline,fancyhdr}
\usepackage{booktabs}
\usepackage{amsmath}
\usepackage{lastpage}
\usepackage[lined,boxed,commentsnumbered]{algorithm2e}
\usepackage{enumerate}
\usepackage{color}
\usepackage{graphicx}							% Standard graphics package
\usepackage{array}
\usepackage{tabularx, caption}
\usepackage{multirow}
\usepackage[framemethod=tikz]{mdframed}% For highlighting paragraph backgrounds
\usepackage{multicol}
\usepackage{rotating}
\usepackage{graphics}
\usepackage{geometry}
\usepackage{setspace}
\usepackage{epsfig}
\usepackage{tikz}
\usepackage{listings}

\usetikzlibrary{arrows,snakes,backgrounds}
\usepackage{hyperref}
\hypersetup{urlcolor=blue,linkcolor=black,citecolor=black,colorlinks=true} 
%\usepackage{pstcol} 								% PSTricks with the standard color package

\newtheorem{theorem}{{\bf Định lý}}
\newtheorem{property}{{\bf Tính chất}}
\newtheorem{proposition}{{\bf Mệnh đề}}
\newtheorem{corollary}[proposition]{{\bf Hệ quả}}
\newtheorem{lemma}[proposition]{{\bf Bổ đề}}

\everymath{\color{blue}}
%\usepackage{fancyhdr}
\setlength{\headheight}{40pt}
\pagestyle{fancy}
\fancyhead{} % clear all header fields
\fancyhead[L]{
 \begin{tabular}{rl}
    \begin{picture}(25,15)(0,0)
    \put(0,-8){\includegraphics[width=8mm, height=8mm]{hcmut.png}}
    %\put(0,-8){\epsfig{width=10mm,figure=hcmut.eps}}
   \end{picture}&
	%\includegraphics[width=8mm, height=8mm]{hcmut.png} & %
	\begin{tabular}{l}
		\textbf{\bf \ttfamily Trường Đại Học Bách Khoa Tp.Hồ Chí Minh}\\
		\textbf{\bf \ttfamily Khoa Khoa Học và Kỹ Thuật Máy Tính}
	\end{tabular} 	
 \end{tabular}
}
\fancyhead[R]{
	\begin{tabular}{l}
		\tiny \bf \\
		\tiny \bf 
	\end{tabular}  }
\fancyfoot{} % clear all footer fields
\fancyfoot[L]{\scriptsize \ttfamily Bài tập lớn môn Mô Hình Hoá Toán Học - Niên khóa 2016-2017}
\fancyfoot[R]{\scriptsize \ttfamily Trang {\thepage}/\pageref{LastPage}}
\renewcommand{\headrulewidth}{0.3pt}
\renewcommand{\footrulewidth}{0.3pt}


%%%
\setcounter{secnumdepth}{4}
\setcounter{tocdepth}{3}
\makeatletter
\newcounter {subsubsubsection}[subsubsection]
\renewcommand\thesubsubsubsection{\thesubsubsection .\@alph\c@subsubsubsection}
\newcommand\subsubsubsection{\@startsection{subsubsubsection}{4}{\z@}%
                                     {-3.25ex\@plus -1ex \@minus -.2ex}%
                                     {1.5ex \@plus .2ex}%
                                     {\normalfont\normalsize\bfseries}}
\newcommand*\l@subsubsubsection{\@dottedtocline{3}{10.0em}{4.1em}}
\newcommand*{\subsubsubsectionmark}[1]{}
\makeatother

\definecolor{dkgreen}{rgb}{0,0.6,0}
\definecolor{gray}{rgb}{0.5,0.5,0.5}
\definecolor{mauve}{rgb}{0.58,0,0.82}
\lstset{frame=tb,
	language=Matlab,
	aboveskip=3mm,
	belowskip=3mm,
	showstringspaces=false,
	columns=flexible,
	basicstyle={\small\ttfamily},
	numbers=none,
	numberstyle=\tiny\color{gray},
	keywordstyle=\color{blue},
	commentstyle=\color{dkgreen},
	stringstyle=\color{mauve},
	breaklines=true,
	breakatwhitespace=true,
	tabsize=3,
	numbers=left,
	stepnumber=1,
	numbersep=1pt,    
	firstnumber=1,
	numberfirstline=true
}

\renewcommand{\chaptername}{Chương}
\renewcommand\bibname{Tài liệu tham khảo}
\usepackage[toc,page]{appendix}
\usepackage{indentfirst}

\begin{document}

\begin{titlepage}
\begin{center}
ĐẠI HỌC QUỐC GIA THÀNH PHỐ HỒ CHÍ  \\
TRƯỜNG ĐẠI HỌC BÁCH KHOA \\
KHOA KHOA HỌC - KỸ THUẬT MÁY TÍNH 
\end{center}

\vspace{1cm}

\begin{figure}[h!]
\begin{center}
\includegraphics[width=3cm]{hcmut.png}
\end{center}
\end{figure}

\begin{center}
\begin{tabular}{c}
	\multicolumn{1}{c}{\textbf{{\Huge LUẬN VĂN TỐT NGHIỆP}}}\\
	~~\\
	\\
	\\
	\hline
	\\
	\multicolumn{1}{l}{\textbf{{\huge Nghiên cứu giải thuật xử lý ảnh}}}\\
	\\
	
	\textbf{{\huge \& Giải pháp Egde Computing tổng thể}}\\
	\\
	\hline
\end{tabular}
\end{center}

\vspace{3cm}

\begin{table}[h]
\begin{tabular}{rrll}
\hspace{5 cm} & GVHD: &TS. LÊ THÀNH SÁCH&\\
& & PGS.TS THOẠI NAM\\
& & \\	
& SV: & Trần Minh Thông & 1413843\\
& & Hồ Bảo Quốc & 1413171 \\
& & Dương Viết Trung & 1414272\\

\end{tabular}
\end{table}

\begin{center}
{\footnotesize TP.HỒ CHÍ MINH, THÁNG 4/2017}
\end{center}
\end{titlepage}


\thispagestyle{empty}

\newpage
\begin{center}
	\textbf{LỜI CAM ĐOAN}
\end{center}

Luận văn của chúng tôi có tham khảo các tài liệu, bài báo, trang web được trình bày ở
mục tài liệu tham khảo và ở mỗi tham khảo chúng tôi đều trích dẫn nguồn gốc. Chúng tôi
xin cam đoan rằng ngoài các trích dẫn từ các tham khảo trên toàn bộ nội dung trong báo cáo
là do chúng tôi tự soạn thảo từ kết quả thực hiện của riêng chúng tôi, không sao chép từ
bất kì tài liệu nào khác.\\

Chúng tôi sẽ hoàn toàn chịu xử lí theo quy định nếu có bất cứ sai phạm nào so với lời
cam kết.

\newpage 
\begin{center}
	\textbf{LỜI CẢM ƠN}
\end{center}

Đầu tiên, nhóm xin gửi lời cảm ơn chân thành đến thầy hướng dẫn TS. Lê Thành Sách
đã luôn trao đổi, góp ý và hướng dẫn chúng tôi trong suốt thời gian qua. Thầy không những
chia sẻ những kiến thức liên quan đến nội dung đề tài mà còn giúp đỡ, định hướng các công 
việc cần thiết và đốc thúc chúng tôi hoàn tất luận văn tốt nghiệp của mình.\\

Xin cảm ơn chân thành PGS.TS Thoại Nam, người đã định hướng đề tài và tin tưởng giao đề 
tài rất hay và thiết thực này cho nhóm. Thầy cũng là người cung cấp thiết bị cũng như 
những sự hỗ trợ cần thiết khác cho nhóm, giúp nhóm có cơ sở để hoàn thành bài tốt nghiệp
 của mình. \\

Xin gửi lời cảm ơn hai anh khóa trên là Minh và Thành  đã chỉ dẫn và hỗ trợ các kiến thức
cùng công cụ nền tảng cần thiết để hiện thực và thử nghiệm chương trình.\\

Xin chân thành cảm ơn đến các thầy cô trường Đại học Bách Khoa thành phố Hồ Chí
Minh và đặc biệt là khoa Khoa học và Kĩ thuật Máy tính đã giảng dạy những kiến thức bổ
ích cho chúng tôi trong suốt thời gian học tại trường.\\

Cuối cùng, chúng tôi gởi lời cảm ơn tới các bạn, người thân đã tham gia giúp đỡ nhóm tạo
nên bộ dữ liệu huấn luyện cho chương trình.\\


\newpage
\begin{center}
	\textbf{TÓM TẮT LUẬN VĂN}
\end{center} 

Trong vài năm trở lại đây, chứng kiến sự bùng nổ của mạng nơ-ron nhân tạo hay phương pháp 
học sâu đã đạt được những thành tựu vượt bậc, đặc biệt trong ngành thị giác máy tính. Các 
ứng dụng của thị giác máy tính len lỏi trong mọi lĩnh vực đời sống của xã hội, nổi bật là 
các ứng dụng tìm kiếm nội dung ảnh, bản đồ, các ứng dụng trong y học, xe tự lái, .. Các ứng
 dụng này đều được xây dựng trên cái bài toán nhỏ hơn như phân loại ảnh (classification), 
 định vị (localization) và phát hiện đối tượng (detection), phân mảnh đối tượng 
 (segmentation).  Trong phạm vi luận văn này, nhóm ưu tiên tìm hiểu bài toán phát hiện
đối tượng, là sự kết hợp giữa hai bài toán định vị đối tượng và phân loại ảnh, sử dụng
phương pháp học sâu.  Cuộc thi nổi tiếng nhất cho bài toán nhận dạng đối tượng là ILSVRC, 
có rất nhiều mô hình dẫn đầu xuất phát từ cuộc thi này. Đặc biệt là mô hình SSD, giành 
chiến thắng cuộc thi ILSVRC \\

Mạng nơ-ron sâu thực chất không phải là một khái niệm mới, mà nó xuất hiện từ những
năm 19xx. Lý do khiến chúng mãi bây giờ mới nổi lên, chính là vì sự lớn nhanh của sức
mạnh phần cứng máy tính cùng trong thời đại bùng nổ về dữ liệu. Các mạng nơ-ron sâu 
đòi hỏi khả năng tính toán vô cùng lớn của phần cứng để tính toán hàng gigabyte,
terabyte dữ liệu. Phần cứng máy tính hiện nay không chỉ có đột phá về sức mạnh tính 
toán mà còn về kích thước của chúng. Hiện nay đã có những máy tính có sức mạnh trung
bình có kích thước chỉ bằng một thẻ bài tây, chính ưu điểm này mở đường cho một xu
hướng mới, tính toán ở cạnh mạng - edge computing, thiết bị tính toán được cài đặt
ngay trên thiết bị quan trắc dữ liệu giúp tối ưu đường truyền, giảm tải cho server,
ận dụng khả năng tính toán. Tuy nhiên, vẫn còn đó những cơ hội cũng như thách thức
đang chờ được giải quyết. Trong phạm vi luận văn, nhóm đã triển khai giải thuật 
phát hiện đối tượng lên thiết bị IoT của Nvidia là Jetson Tx2. 

\newpage
\begin{center}
	\textbf{BỐ CỤC LUẬN VĂN}
\end{center} 

Luận văn gồm 3 phần chính, phần mở đầu, phần hiện thực và kết luận \\

Phần mở đầu được nêu trong CHƯƠNG 1: GIỚI THIỆU, nhằm cung cấp cái nhìn tổng quan về đề tài,
 mục tiêu của nhóm cũng như quá trình nghiên cứu, thực hiện luận văn của nhóm.\\

Phần hiện thực gồm 4 chương\\
\begin{itemize}
	\item CHƯƠNG 2: KIẾN THỨC NỀN TẢNG. Trình bày kiến thức cơ sở của bài toán phát hiện
	 đối tượng, ưu điểm của hướng tiếp cận sử dụng mạng nơ-ron tích chập. Một số bài 
	 toán và mô hình liên quan trực tiếp tới bài toán phát hiện đối tượng. Trình bày 
	 những kiến thức cơ bản về tính toán cạnh mạng, các thông số cần biết của thiết bị
	  và nêu lên những cơ hội cũng như khó khăn còn tồn tại của tính toán mạng. 
	\item CHƯƠNG 3: CÁC CÔNG TRÌNH LIÊN QUAN. Trình bày một số một hình liên quan 
	trực tiếp và được sử dụng trong luận văn, chủ yếu là mô hình ssd và các biến thể 
	của nó.

	\item CHƯƠNG 4: NỘI DUNG THỰC HIỆN
\end{itemize}

Phần kết luận là chương cuối cùng, CHƯƠNG 5: KẾT LUẬN, ĐÁNH GIÁ, HƯỚNG PHÁT TRIỂN. Tại đây, nhóm xin đưa
ra một số nhận xét về công trình nghiên cứu của mình, cũng như những đóng góp mà luận
văn đem lại. Bên cạnh đó, nhóm nhìn nhận lại những sai sót và đưa ra những hướng giải
quyết, phát triển trong tương lai.\\

Ngoài 3 phần chính vừa nêu trên, luận văn còn bao gồm N phụ lục bổ sung thêm các
thông tin cần thiết nhằm hỗ trợ các nhóm phát triển sau. Cụ thể:
\begin{itemize}
	\item PHỤ LỤC 1:
\end{itemize}

\newpage %danh muc bang
\listoftables
\newpage
\listoffigures


\newpage
\tableofcontents
\newpage

%%%%%%%%%%%%%%%%%%%%%%%%%%%%%%%%%


%%%%%%%%%%%%%%%%%%%%%%%%%%%%%%%%%
\chapter{GIỚI THIỆU} 
Trong phần thứ nhất của luận văn, nhóm sẽ 
trình bày nguyên nhân nhóm chọn đề tài cũng như mục tiêu mà nhóm mong muốn đạt được 
khi thực hiện đề tài.

\section{Lý do chọn đề tài}

	Trong những năm gần đây, những nghiên cứu về deep learning đang thu được
những kết quả ấn tượng, đặc biệt trong lĩnh vực thị giác máy tính, các mô 
hình deep learning thu được kết quả có độ chính xác cao hơn và có nhiều ưu 
điểm vượt trội so với các
phương pháp thị giác máy tính truyền thống. Chúng đã đưa các công nghệ tương lai
 như xe tự lái, hệ thống giám sát theo dõi qua video, nhận diện khuôn mặt, .. đến 
 gần với hiện thực hơn, hiệu quả hơn và thậm chí một số công nghệ đã được áp dụng 
 thành công trong một số điều kiện nhất định. Sự thành công đã được kiểm chứng và 
 tiềm năng to lớn của deep learning, nhóm mong muốn có thể áp dụng những giải thuật, 
 những mô hình hiện có và tìm cách cải tiến nó để giải quyết những vấn đề thực tế ở Việt Nam hiện nay. \\
 
 Cuối năm 2017, phóng viên của đài BBC đã thực hiện một đoạn phóng sự về hệ thống giám sát qua video tại thành phố Quý Dương,
 của Trung Quốc \footnote{https://techcrunch.com/2017/12/13/china-cctv-bbc-reporter/}, hệ thống này sử dụng một mạng lưới dày 
 đặc các camera được tích hợp trí tuệ nhận tạo và xây dựng một cơ sở dữ liệu của người dân thành phố. Trong phóng sự, hệ thống 
 này sau khi nhập dữ liệu mới là nhân dạng khuôn mặt của phóng viên BBC, thì nó cần hơn 7
  phút để tìm ra vị trí của phóng viên trong thành phố. Đây là một trong nhiều ứng dụng đáng kinh ngạc 
  của trí tuệ nhân tạo. Sự mới mẻ mang tính đột phá chính là động lực cho nhóm thực hiện đề tài này.

\section{Tính thực tiễn}
Một số vấn đề thực tế và khả năng ứng dụng của đề tài:
\begin{itemize}
\item Phát hiện ngủ gật. Hằng năm ở nước ta vẫn tồn tại những vụ tai nạn thương tâm xảy 
ra do tài xế ngủ gật. Một hệ thống phát hiện trạng thái của lái xe có thể giảm thiểu khả 
năng xảy tai nạn thảm khốc hằng năm. Một hệ thống được lắp đặt trên xe và đủ thông minh 
để xử lí các tình huống khác nhau, các điều kiện khác nhau trong thực tế. 

\item Tìm kiếm nội dung trong video. Một đứa bé thất lạc ở một trung tâm thương mại lớn,
 hệ thống camera     
\item Giám sát giao thông
\end{itemize}
\section{Mục tiêu của luận văn}
\section{Quy trình thực hiện luận văn}

\chapter{KIẾN THỨC NỀN TẢNG}
\section{Kiến thức xử lý ảnh}
\subsection{Bài toán phát hiện đối tượng}
\subsection{Phương pháp thị giác máy tính truyền thống}
\subsection{Phương pháp học sâu}
\section{Điện toán cạnh mạng}

\chapter{CÁC CÔNG TRÌNH LIÊN QUAN }
\section{Single Shot MultiBox Detector}
\section{Feature Fusion Single Shot MultiBox Detector}
\section{Single Shot Scale-invariant Face Detector }

\chapter{NỘI DUNG THỰC HIỆN}
\section{Mục tiêu} thử nghiệm các mô hình và giải pháp cải tiến, sau đó triển khai lên computing box
\section{Thử nghiệm các mô hình}
\subsection{Chuẩn bị dữ liệu}
\subsection{Một số tham số cần lưu ý}
\subsection{Đánh giá mô hình}
\subsection{Tối ưu hàm mất mát}

\section{Triển khai mô hình lên board Jetson}

\chapter{KẾT LUẬN, ĐÁNH GIÁ, HƯỚNG PHÁT TRIỂN}
%%%chú thích 
<abc>
\footnote{là abcd}

%%%insert ảnh, nhớ captioning và ghi nguồn 
%-------------Figure 2 ---------------------------------
%\begin{figure}[h!]
%	\centering
%	\includegraphics[width=0.8\textwidth]{anm_victim_destop2.PNG}
%	\caption{Tình trạng máy nạn nhân sau cuộc tấn công}
%	\end{figure}
	
%%%%%%%%%%%%%%%%%


%{PHỤ LỤC}
%%%%%%%%%%%%%%%%%%%%%%%%%%%%%%%%%
%\addcontentsline{\bibitem{adf}}

\begin{thebibliography}{80}
\addcontentsline{toc}{chapter}{Tài liệu tham khảo}

\bibitem{ssd}
Wei Liu, Dragomir Anguelov, Dumitru Erhan, Christian Szegedy, Scott Reed, Cheng-Yang Fu, Alexander C. Berg, "SSD: Single Shot MultiBox Detector", ECCV 2016

\bibitem{fssd}
Zuoxin Li, Fuqiang Zhou, "FSSD: Feature Fusion Single Shot Multibox Detector", arXiv 2017

\bibitem{s3sd} Shifeng Zhang, Xiangyu Zhu, Zhen Lei, Hailin Shi, Xiaobo Wang, Stan Z.Li, "S$^ 3$FD: Single Shot Scale-invariant Face Detector", ICCV 2017

\bibitem{fasterrcnn} Shaoqing Ren and Kaiming He and Ross Girshick and Jian Sun, "Faster {R-CNN}: Towards Real-Time Object Detection
with Region Proposal Networks", NIPS 2015

\bibitem{fastrcnn} Ross Girshick, "Fast{R-CNN}", ICCV 2015

\bibitem{rcnn} Ross Girshick, Jeff Donahue, Trevor Darrell, Jitendra Malik, "Rich feature hierarchies for accurate object detection and semantic segmentation", CVPR 2014 

\bibitem{dssd} RCheng-Yang Fu, Wei Liu, Ananth Ranga, Ambrish Tyagi, Alexander C. Berg, "DSSD : Deconvolutional Single Shot Detector", CVPR 2014 

\bibitem{CVX}
CVX Introduction
``\textbf{link: http://cvxr.com/cvx/doc/intro.html/}'',
\textit{What is CVX}, lần truy cập cuối: 15/04/2017.

\end{thebibliography}
\end{document}

