\documentclass[a4paper]{report}
\usepackage{vntex}
%\usepackage[english,vietnam]{babel}
%\usepackage[utf8]{inputenc}

%\usepackage[utf8]{inputenc}
%\usepackage[francais]{babel}
\usepackage{a4wide,amssymb,epsfig,latexsym,multicol,array,hhline,fancyhdr}
\usepackage{booktabs}
\usepackage{amsmath}
\usepackage{lastpage}
\usepackage[lined,boxed,commentsnumbered]{algorithm2e}
\usepackage{enumerate}
\usepackage{color}
\usepackage{graphicx}							% Standard graphics package
\usepackage{array}
\usepackage{tabularx, caption}
\usepackage{multirow}
\usepackage[framemethod=tikz]{mdframed}% For highlighting paragraph backgrounds
\usepackage{multicol}
\usepackage{rotating}
\usepackage{graphics}
\usepackage{geometry}
\usepackage{setspace}
\usepackage{epsfig}
\usepackage{tikz}
\usepackage{listings}
\usetikzlibrary{arrows,snakes,backgrounds}
\usepackage{hyperref}
\hypersetup{urlcolor=blue,linkcolor=black,citecolor=black,colorlinks=true} 
%\usepackage{pstcol} 								% PSTricks with the standard color package

\newtheorem{theorem}{{\bf Định lý}}
\newtheorem{property}{{\bf Tính chất}}
\newtheorem{proposition}{{\bf Mệnh đề}}
\newtheorem{corollary}[proposition]{{\bf Hệ quả}}
\newtheorem{lemma}[proposition]{{\bf Bổ đề}}

\everymath{\color{blue}}
%\usepackage{fancyhdr}
\setlength{\headheight}{40pt}
\pagestyle{fancy}
\fancyhead{} % clear all header fields
\fancyhead[L]{
 \begin{tabular}{rl}
    \begin{picture}(25,15)(0,0)
    \put(0,-8){\includegraphics[width=8mm, height=8mm]{hcmut.png}}
    %\put(0,-8){\epsfig{width=10mm,figure=hcmut.eps}}
   \end{picture}&
	%\includegraphics[width=8mm, height=8mm]{hcmut.png} & %
	\begin{tabular}{l}
		\textbf{\bf \ttfamily Trường Đại Học Bách Khoa Tp.Hồ Chí Minh}\\
		\textbf{\bf \ttfamily Khoa Khoa Học và Kỹ Thuật Máy Tính}
	\end{tabular} 	
 \end{tabular}
}
\fancyhead[R]{
	\begin{tabular}{l}
		\tiny \bf \\
		\tiny \bf 
	\end{tabular}  }
\fancyfoot{} % clear all footer fields
\fancyfoot[L]{\scriptsize \ttfamily Bài tập lớn môn Mô Hình Hoá Toán Học - Niên khóa 2016-2017}
\fancyfoot[R]{\scriptsize \ttfamily Trang {\thepage}/\pageref{LastPage}}
\renewcommand{\headrulewidth}{0.3pt}
\renewcommand{\footrulewidth}{0.3pt}


%%%
\setcounter{secnumdepth}{4}
\setcounter{tocdepth}{3}
\makeatletter
\newcounter {subsubsubsection}[subsubsection]
\renewcommand\thesubsubsubsection{\thesubsubsection .\@alph\c@subsubsubsection}
\newcommand\subsubsubsection{\@startsection{subsubsubsection}{4}{\z@}%
                                     {-3.25ex\@plus -1ex \@minus -.2ex}%
                                     {1.5ex \@plus .2ex}%
                                     {\normalfont\normalsize\bfseries}}
\newcommand*\l@subsubsubsection{\@dottedtocline{3}{10.0em}{4.1em}}
\newcommand*{\subsubsubsectionmark}[1]{}
\makeatother

\definecolor{dkgreen}{rgb}{0,0.6,0}
\definecolor{gray}{rgb}{0.5,0.5,0.5}
\definecolor{mauve}{rgb}{0.58,0,0.82}
\lstset{frame=tb,
	language=Matlab,
	aboveskip=3mm,
	belowskip=3mm,
	showstringspaces=false,
	columns=flexible,
	basicstyle={\small\ttfamily},
	numbers=none,
	numberstyle=\tiny\color{gray},
	keywordstyle=\color{blue},
	commentstyle=\color{dkgreen},
	stringstyle=\color{mauve},
	breaklines=true,
	breakatwhitespace=true,
	tabsize=3,
	numbers=left,
	stepnumber=1,
	numbersep=1pt,    
	firstnumber=1,
	numberfirstline=true
}
\renewcommand{\chaptername}{Chương}
\renewcommand\bibname{Tài liệu tham khảo}
\begin{document}

\begin{titlepage}
\begin{center}
ĐẠI HỌC QUỐC GIA THÀNH PHỐ HỒ CHÍ MINH \\
TRƯỜNG ĐẠI HỌC BÁCH KHOA \\
KHOA KHOA HỌC - KỸ THUẬT MÁY TÍNH 
\end{center}

\vspace{1cm}

\begin{figure}[h!]
\begin{center}
\includegraphics[width=3cm]{hcmut.png}
\end{center}
\end{figure}

\vspace{1cm}


\begin{center}
\begin{tabular}{c}
	\multicolumn{1}{l}{\textbf{{\Large MÔ HÌNH HOÁ TOÁN HỌC}}}\\
	~~\\
	\hline
	\\
	\multicolumn{1}{l}{\textbf{{\Large Xử lý tối ưu bài toán}}}\\
	\\
	
	\textbf{{\Huge Hospitals \& Residents trên Matlab}}\\
	\\
	\hline
\end{tabular}
\end{center}

\vspace{3cm}

\begin{table}[h]
\begin{tabular}{rrl}
\hspace{5 cm} & GVHD: &Lê Hồng Trang\\
& SV: & Chìu Tuấn Bình - 1510221\\
& & Mai Đức Tú - 1513924 \\
& & Phồng Quang Tuấn - 1513865\\
& & Lê Duy Hiển - 1511057 \\
& & Nguyễn Đỗ Đức Anh - 1510062\\
\end{tabular}
\end{table}

\begin{center}
{\footnotesize TP.HỒ CHÍ MINH, THÁNG 4/2017}
\end{center}
\end{titlepage}


\thispagestyle{empty}

\newpage
\begin{center}
	LỜI CAM ĐOAN
\end{center}

Luận văn của chúng tôi có tham khảo các tài liệu, bài báo, trang web được trình bày ở
mục tài liệu tham khảo và ở mỗi tham khảo chúng tôi đều trích dẫn nguồn gốc. Chúng tôi
xin cam đoan rằng ngoài các trích dẫn từ các tham khảo trên toàn bộ nội dung trong báo cáo
là do chúng tôi tự soạn thảo từ kết quả thực hiện của riêng chúng tôi, không sao chép từ
bất kì tài liệu nào khác.\\

Chúng tôi sẽ hoàn toàn chịu xử lí theo quy định nếu có bất cứ sai phạm nào so với lời
cam kết.

\newpage 
\begin{center}
	LỜI CẢM ƠN
\end{center}

\newpage
\begin{center}
	TÓM TẮT LUẬN VĂN
\end{center} 

\newpage
\begin{center}
	BỐ CỤC LUẬN VĂN
\end{center} 

\newpage
\tableofcontents
\newpage

%%%%%%%%%%%%%%%%%%%%%%%%%%%%%%%%%


%%%%%%%%%%%%%%%%%%%%%%%%%%%%%%%%%
\chapter{GIỚI THIỆU}
\section{Lý do chọn đề tài}
\section{Tính thực tiễn}
\section{Mục tiêu của luận văn}
\section{Qui trình thực hiện luận văn}

\chapter{KIẾN THỨC NỀN TẢNG}
\section{Kiến thức xử lý ảnh}
\subsection{Bài toán phát hiện đối tượng}
\subsection{Phương pháp thị giác máy tính truyền thống}
\subsection{Phương pháp học sâu}
\section{Điện toán cạnh mạng}

\chapter{CÁC CÔNG TRÌNH LIÊN QUAN }
\section{Single Shot MultiBox Detector}
\section{Feature Fusion Single Shot MultiBox Detector}
\section{Single Shot Scale-invariant Face Detector }

\chapter{NỘI DUNG THỰC HIỆN}
\section{Mục tiêu} thử nghiệm các mô hình và giải pháp cải tiến, sau đó triển khai lên computing box
\section{Thử nghiệm các mô hình}
\subsection{Chuẩn bị dữ liệu}
\subsection{Một số tham số cần lưu ý}
\subsection{Đánh giá mô hình}
\subsection{Tối ưu hàm mất mát}

\section{Triển khai mô hình lên board Jetson}

\chapter{KẾT LUẬN, ĐÁNH GIÁ, HƯỚNG PHÁT TRIỂN}

\chapter*{PHỤ LỤC}
%%%%%%%%%%%%%%%%%%%%%%%%%%%%%%%%%
\begin{thebibliography}{80}

\bibitem{CVX}
CVX Introduction
``\textbf{link: http://cvxr.com/cvx/doc/intro.html/}'',
\textit{What is CVX}, lần truy cập cuối: 15/04/2017.

\end{thebibliography}
\end{document}

